%% ---------------------------------------------------------------------------
%% intro.tex
%%
%% Introduction
%%
%% $Id: intro.tex 1477 2010-07-28 21:34:43Z palvarado $
%% ---------------------------------------------------------------------------

\chapter{Introducción}
\label{chp:intro}

En las últimas décadas se han desarrollado amplias investigaciones entorno al área de visión artificial(también conocida como visión por computador CV-Computer Vision)[], con el fin de imitar las características del sistema visual humano tales como: acomodación, adaptación y discriminación ocular []; y poder aplicarlas a actividades como: análisis de imágenes medicas, monitorización de señales de transito y rutas, vigilancia antirrobo y pruebas forenses[], videojuegos interactivos y automatización de procesos industriales. 

El área del sistema embebidos(Embedded Systems), también se han realizado grandes avances para el desarrollo sistemas con una gran capacidad de procesamiento de información a un bajo costo, actualmente los sistemas embebidos esta son omnipresentes diversas aplicaciones como: la automatización industrial, en productos de consumo, comunicaciones y desarrollos militares. Por esto se estima que para el 2014, el mercado de la tecnología integrada asciendará a 142,4 mil millones de dólares y se espera un crecimiento sostenido de un 7 anual por los próximos 5 años[1].

%Los sistemas empotrados son omnipresentes: los encuentran en los automóviles, aparatos de cocina, dispositivos de electrónica de consumo, equipos médicos, y un sinnúmero de otros lugares.


A finales de la década pasada(2009) nace el proyecto Alianza de Visión Embebida (EVA-Embedded Vision Alliance) con la idea de unificar estas dos tecnologías: visión artificial y sistemas embebidos. El objetivo principal de esta alianza es inspirar y empoderar a los diseñadores de sistemas embebidos y desarrolladores de aplicaciones a utilizar tecnología de visión integrada [2]. 

Debido al desarrollo de esta nueva tendencia de Visión Embebida, en el Instituto Tecnológico de Costa Rica, específicamente en el Laboratorio de Procesamiento de Señales e Imágenes (Sip-Lab)[4], nace la iniciativa de incorporarse en esta nueva área, por esto se empiezan a adquirir nuevos equipos y tarjetas de desarrollo en Visión Embebida. El primer equipo en obtenerse fue el Blackfin Embedded Vision Starter Kit, que es una plataforma de hardware versátil, con las herramientas de desarrollo de software necesarias para permitir la construcción de alto rendimiento; esta basado en un procesador Blackfin BF609 de bajo costo, el cual incluye un "procesador de canalización de visión" que simplifica y acelera los algoritmos de procesamiento de imágenes[3]. A pesar de toda la capacidad de esta plataforma, actualmente no se ofrece datos reales sobre el rendimiento y velocidad de procesamiento.

Un objetivo del Sip-Lab es, desarrollar aplicaciones en tiempo real con un bajo consumo energético, por esto es de vital importante para el laboratorio, tener información veraz sobre la capacidad de procesamiento y el consumo energético del Blackfin Embedded Vision Starter Kit ante diferentes escenarios en la utilización del hardware disponible en ella, pues esta información servirá como insumo a eventuales aplicaciones desarrollados por el laboratorio.

  





%%1-http://www.bccresearch.com/market-research/information-technology/embedded-systems-technologies-markets-ift016d.html
%%2-http://www.embedded-vision.com/the-ev-alliance
%%3-http://www.finboard.org/content/blackfin%C2%AE-embedded-vision-starter-kit
%%4-http://www.ie.itcr.ac.cr/siplab/index.php/Main/HomePage

\section{Objetivos y estructura del documento}

\index{objetivos}
Esta plantilla LaTeX tiene como objetivo simplificar la construcción del
documento de tesis, presentando ejemplo de figuras y tablas, así como otorgar
una plataforma de compilación en GNU/Linux que simplifique la administración de
todo el documento.

La última sección de la introducción usualmente sí tiene un título estandar que
es ``Objetivos y estructura del documento'', donde se presentan \emph{en prosa}
los objetivos general y específicos que ha tenido el proyecto de graduación,
así como la estructura de la tesis (por ejemplo, ``en el siguiente capítulo se
esbozan los fundamentos teóricos necesarios para explicar en el
capítulo~\ref{ch:solucion} la propuesta realizada$\ldots$''

%%% Local Variables: 
%%% mode: latex
%%% TeX-master: "main"
%%% End: 
